In the first three runs, the current is very close in strength. This shows that the axial field is uniform in strength across the coil. Overall, our data indicates that the direction of the magnetic field generated by the coil points upward, indicating that the current runs counter-clockwise around the coil. This is supported by the construction of the circuit, in which the positive terminal of the signal generator is connected to the coil on the right side, and the negative terminal is connected to the left side of the coil.

\bigskip
The direction of the current is reversed when the terminals are switched, so it follows that the axial peak amplitude of the \textit{East, rev} run is negative while the others are positive. In that run, the current runs clockwise in the coil and the magnetic field points downward. This difference in current flow direction between the \textit{West} and \textit{East, rev.} runs (and similarly, the reversed polarity of the magnetic field between the two) explains the difference in the graphs of the radial field strength with respect to the position of the sensor probe.

\bigskip
The calculated bound values using Equations \ref{single_coil} and \ref{long_solenoid} were both higher than the measured axial peak amplitudes. The measured amplitudes were all closer to that of a short solenoid than to that of a long solenoid.

\bigskip
Due to the symmetry of the system, the radial component of the field should be zero. However, since we were measuring the magnetic field of the coil at its ends, the existence of a non-zero radial component shows the behavior of the magnetic field past the ends of the coil. Based on our data, it appears that the field, as represented by magnetic field lines, curves in toward the center of the coil as the lines enter the coil, and out away from the center of the coil as the lines leave the coil. This explains why the radial field in the \textit{East} and \textit{West} looks the way that it does.

\bigskip
The appearance of a non-zero radial component in the \textit{Center} run is due in large part to error. While removing the sensor probe from the center of the coil, we allowed the probe to move slightly in a direction perpendicular to the vertical axis of the coil. This introduced a source of experimental error, which does not necessarily show the existence of a non-zero radial component for that run.