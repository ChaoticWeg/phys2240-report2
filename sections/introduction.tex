In this lab, we will move a magnetic field sensor through a current-carrying coil and record data about the magnetic field in order to compare changes in axial and radial components of the field. We will attach the magnetic field sensor to a rotary motion sensor in order to record the position of the magnetic field sensor relative to the coil.

\bigskip
When referring to the magnetic field created by a solenoid, we call the component that runs parallel to the vertical axis of the solenoid the \textbf{axial component} of the field; likewise, we call the component that runs perpendicular to the vertical axis of the solenoid the \textbf{radial component} of the field. The direction of each field component, as measured by the sensor, is relative to the orientation of the sensor. This direction is shown with positive and negative values.

\bigskip
The general equation for finding the magnetic field along the perpendicular axis through the center of a coil of wire with negligible length, radius $R$, and $N$ turns of wire (\textbf{Equation \ref{single_coil}}) is

\begin{equation} \label{single_coil}
B = \frac{\mu_0 N I R^2}{2 ( x^2 + R^2 )^{3/2}}
\end{equation}

where \textbf{$\mu_0$} $= 4 \pi \times 10^{-7}$ $T \cdot \frac{m}{A}$ is the permittivity of free space, \textbf{$I$} is the current through the coil, and \textbf{$x$} is the distance from the center of the coil. To find the magnetic field of a long solenoid with $n$ turns per unit length, we use \textbf{Equation \ref{long_solenoid}}:

\begin{equation} \label{long_solenoid}
B = \mu_0 n I
\end{equation}

where $\mu_0$ is again the permittivity of free space, shown above. Equation \ref{long_solenoid} fails when approaching the ends of the solenoid, where the magnetic field strength begins to decrease.

\bigskip
To be considered a \textit{long} solenoid, the length of the coil inside the solenoid must be significantly longer than the diameter of the coil. Solenoids that do not fit this description are referred to as \textit{short} solenoids. For short solenoids, we are not able to use either Equation \ref{single_coil} because the length of the coil is too long, and we are not able to use Equation \ref{long_solenoid} because the length of the coil is too short. Instead, the two equations provide bounds on the value of the magnetic field.

\bigskip
\begin{minipage}{\textwidth}
    \centering
    Specifications for the Solenoid \\
    \medskip
    \begin{tabular}{| l | r |} \hline
    
    $N$  &  600 turns \\ \hline
    $R$  &  0.015 m   \\ \hline
    $L$  &  0.025 m   \\ \hline
    
\end{tabular}
\end{minipage}