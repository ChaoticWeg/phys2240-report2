We first zeroed the magnetic field sensor and the rotary motion sensor without any current running through the solenoid. We ran 5V of DC current through the coil, inserted the magnetic field sensor as far as possible into the coil (while centered about the vertical axis of the coil), and recorded the data from the magnetic sensor probe as we slowly pulled the sensor out of the coil with the handle facing north. We labeled this data set \textbf{Center}. We measured and recorded the peak amplitude.

\bigskip
We repeated this process with the probe held against the \textbf{East} wall of the coil, and again with the probe held against the \textbf{West} wall of the coil. We switched the patch cords, reversing the direction of the current through the coil and repeated the process with the probe held against the east wall of the coil. We labeled this data set \textbf{East, rev}.

\bigskip
We then calculated the bound values for our solenoid using Equations \ref{single_coil} and \ref{long_solenoid}.
